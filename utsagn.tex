\documentclass[11pt]{article}
\usepackage{pstricks,pst-plot}
\usepackage[utf8]{inputenc}
\usepackage[english,norsk]{babel}

\usepackage{listings}
\usepackage{color}
\usepackage{xcolor}
\usepackage{marvosym}



\lstset{ %
  backgroundcolor=\color{white},   % choose the background color; you must add \usepackage{color} or \usepackage{xcolor}
  basicstyle=\footnotesize,        % the size of the fonts that are used for the code
  breakatwhitespace=false,         % sets if automatic breaks should only happen at whitespace
  breaklines=true,                 % sets automatic line breaking
  %captionpos=b,                   % sets the caption-position to bottom
  commentstyle=\color{blue},       % comment style
  deletekeywords={...},            % if you want to delete keywords from the given language
  escapeinside={\%*}{*)},          % if you want to add LaTeX within your code
  extendedchars=true,              % lets you use non-ASCII characters; for 8-bits encodings only, does not work with UTF-8
  keepspaces=true,                 % keeps spaces in text, useful for keeping indentation of code (possibly needs columns=flexible)
  keywordstyle=\color{orange},       % keyword style
  language=Python,                 % the language of the code
  morekeywords={*,...},            % if you want to add more keywords to the set
  rulecolor=\color{black},         % if not set, the frame-color may be changed on line-breaks within not-black text (e.g. comments (green here))
  showspaces=false,                % show spaces everywhere adding particular underscores; it overrides 'showstringspaces'
  showstringspaces=false,          % underline spaces within strings only
  showtabs=false,                  % show tabs within strings adding particular underscores
  stepnumber=2,                    % the step between two line-numbers. If it's 1, each line will be numbered
  stringstyle=\color{violet},      % string literal style
  tabsize=2,                       % sets default tabsize to 2 spaces
}

\usepackage{caption}
\DeclareCaptionFont{white}{\color{white}}
\DeclareCaptionFormat{listing}{\colorbox{gray}{\parbox{\textwidth}{#1#2#3}}}
\captionsetup[lstlisting]{format=listing,labelfont=white,textfont=white}

% This concludes the preamble

\usepackage[colorlinks=true,citecolor=red]{hyperref}

\usepackage{amsmath, amssymb, amsthm}
\usepackage{makeidx}
\usepackage{graphicx}


\usepackage{accents}
\newcommand{\interior}[1]{\accentset{\smash{\raisebox{-0.12ex}{$\scriptstyle\circ$}}}{#1}\rule{0pt}{2.3ex}}
\fboxrule0.0001pt \fboxsep0pt

% used for diagrams, not needed here
% \usepackage{tikz}
% \usepackage{tikz-cd}
% \usetikzlibrary{matrix, arrows, decorations}

\usepackage[natbib]

%%% New environments
\newtheorem{theorem}{Theorem}[section]
\newtheorem{lemma}[theorem]{Lemma}
\newtheorem{corollary}[theorem]{Corollary}
\newtheorem{prop}[theorem]{Proposition}

\theoremstyle{definition}
\newtheorem{defn}[theorem]{Definition}
\newtheorem{aksiom}[theorem]{Aksiom}
\newtheorem{exercise}[theorem]{Exercise}
\newtheorem{oppgave}[theorem]{Oppgave}
\newtheorem{problem}[theorem]{Problem}
\newtheorem{remark}[theorem]{Remark}
\newtheorem{nremark}[theorem]{Bemerkning}
\newtheorem{question}[theorem]{Question}
\newtheorem{conjecture}[theorem]{Conjecture}
\newtheorem{improvement}[theorem]{Improvement}
\newtheorem{discus}[theorem]{Discus}
\newtheorem{ptheorem}[theorem]{Possible Theorem}
\newtheorem{project}[theorem]{Project}
\newtheorem{solution}[theorem]{Solution}
%\newtheorem{example}[theorem]{Example}
%\newtheorem{eksempel}[theorem]{Eksempel}

\usepackage{thmtools}
\theoremstyle{definition}
\declaretheorem[name=Eksempel,qed={\lower-0.3ex\hbox{\Smiley}}]{eksempel}

\declaretheorem[name=eksempel,qed={\lower-0.3ex\hbox{\frowney}}]{deksempel}

\usepackage{skull}
\declaretheorem[name=moteksempel,qed={\lower-0.3ex\hbox{\skull}}]{moteks}



\newcommand\hra{\hookrightarrow}

%%% Custom definitions and macros
\def\beq{\begin{equation}}
\def\eeq{\end{equation}}

\newcommand{\Id}{{\bf 1}}

\DeclareMathOperator{\vspan}{Span}
\renewcommand{\d}{\mathrm{\; d}} % for differensialet i integraler.

\newcommand{\Aut}{\mathop{{\rm Aut}}}
\newcommand{\End}{\mathop{{\rm End}}}
\newcommand{\Hom}{\mathop{{\rm Hom}}}
\newcommand{\rank}{\mathop{{\rm rank}}}
\renewcommand{\div}{\mathop{{\rm div}}}

\DeclareMathOperator{\Dx}{\frac{\d}{\d x}}
\DeclareMathOperator{\DDx}{\frac{\d^2}{\d x^2}}
\DeclareMathOperator{\erf}{erf}
\DeclareMathOperator{\erfc}{erfc}


% Caligraphic letters
\newcommand{\cA}{\mathcal{A}}
\newcommand{\cB}{\mathcal{B}}
\newcommand{\cR}{\mathcal{R}}
\newcommand{\cC}{\mathcal{C}}
\newcommand{\cD}{\mathcal{D}}
\newcommand{\cE}{\mathcal{E}}
\newcommand{\cF}{\mathcal{F}}
\newcommand{\cG}{\mathcal{G}}
\newcommand{\cH}{\mathcal{H}}
\newcommand{\cI}{\mathcal{I}}
\newcommand{\cJ}{\mathcal{J}}
\newcommand{\cK}{\mathcal{K}}
\newcommand{\cL}{\mathcal{L}}
\newcommand{\cN}{\mathcal{N}}
\newcommand{\cO}{\mathcal{O}}
\newcommand{\cS}{\mathcal{S}}
\newcommand{\cZ}{\mathcal{Z}}
\newcommand{\cP}{\mathcal{P}}
\newcommand{\cT}{\mathcal{T}}
\newcommand{\cU}{\mathcal{U}}
\newcommand{\cV}{\mathcal{V}}
\newcommand{\cX}{\mathcal{X}}
\newcommand{\cW}{\mathcal{W}}

\newcommand{\A}{\mathbb{A}}
\newcommand{\B}{\mathbb{B}}
\newcommand{\C}{\mathbb{C}}
\newcommand{\D}{\mathbb{D}}
\renewcommand{\H}{\mathbb{H}}
\newcommand{\N}{\mathbb{N}}
\newcommand{\K}{\mathbb{K}}
\newcommand{\Q}{\mathbb{Q}}
\newcommand{\Z}{\mathbb{Z}}
\renewcommand{\P}{\mathbb{P}}
\newcommand{\R}{\mathbb{R}}
\newcommand{\U}{\mathbb{U}}
\newcommand{\bT}{\mathbb{T}}
\newcommand{\bD}{\mathbb{D}}

\newcommand{\vu}{\boldsymbol{u}}
\newcommand{\vv}{\boldsymbol{v}}
\newcommand{\vn}{\boldsymbol{n}}
\newcommand{\vf}{\boldsymbol{f}}

\def\di{\partial}
\def\bs{\backslash}
\def\e{\epsilon}
\def\la{\langle}
\def\ra{\rangle}

\pagestyle{empty}

%\makeindex

\title{Om matematiske utsagn}
\author{Eindride Kjersheim}

\begin{document}

\maketitle

\begin{abstract}
Notatet utdyper noen temaer fra matematikk 1T i den videregående skolen.
Flere av disse er også fremstilt som guidede oppgaver. Noen temaer er lagt fram mer
nøye/teoretisk, der vi går steg for steg fra aksiom til formel. Dette er
for å se hvor regnereglene kommer fra.
\end{abstract}

\section{Ulikheter}
For reelle tall kan vi alltid sammenlikne to tall, og bestemme
hvilket tall som er størst, (eller minst). F.eks om du skal kjøpe
en spesiell brus og det er to butikker ved siden av hverandre,
   handler du helst der hvor brusen er billigst. Om den ene butikken
   tar $13$ kroner og den andre $14$, velger du butikken hvor den koster
   $13$ kroner fordi 
   \[13 < 14,\] 
   altså at $13$ er mindre enn $14$.
   Man kan gjerne huske dette ved å tenke seg at ulikhetstegnet er
   en krokodillemunn $<$ som snur og glefser mot det tallet som er størst.
   Dersom tallene kan være like store, er dette det samme som at vår krokodille
   ikke klarer å
   bestemme seg helt så vi kan skrive $13\leq 13$, men også
   $13 \leq 14$. Dette er spesielt nyttig når vi regner med uttrykk
   hvor vi f.eks tillater at et tall er større enn et annet, men også
   at det er likt.
\subsection{Sannhetsverdi, utsagn}
Merk at om vi tar to tall $x, y \in \R$ og tester $x \leq y$ får
vi enten sant eller usant. Enten er $x \leq y$, eller så er
$y \leq x$, men aldri begge samtidig med mindre $x = y$.
\begin{eksempel}
$13 < 14$ er sant, men $14 < 13$  er \textbf{usant}
\end{eksempel}

Vi kan også bruke dette for å beskrive mengder av tall:

\begin{eksempel}
$x\leq 2$  er sant for alle tall $x$ som er mindre eller lik $2$,
    og \textbf{usant} for alle tall $x$ som er strengt større enn 2.
Du setter altså inn tall for $x$  og tester om du får sant eller ikke.
\emph{Mengden} av alle tall som tilfredstiller vår ulikhet, kaller vi
$\la \leftarrow, 2]$. Du må gjerne tenke på mengder som samlinger av
objekter. I vårt tilfelle er $\la \leftarrow, 2]$ en samling av tall.
\end{eksempel}

\subsection{Oppgaver}

\begin{oppgave}
La $x \leq 2$. Marker sant og usant for ulike verdier av $x$
\begin{itemize}
\item[i)]
$x = 1$
\item[ii)]
$x = 10$
\item[iii)]
$x = 2$
\item[iv)]
$x = 7$
\item[v)]
$x = 0$
\item[vi)]
$x = -100$
\item[vii)]
$x = \pi$
\item[viii)]
$x = \sqrt{2},$  begrunn svaret.
\item[iix)]
$x = 2\sqrt{2},$  begrunn svaret.
\item[ix)]
$x = \frac{5}{2},$  begrunn svaret.
\item[x)]
$x = \frac{5\sqrt{2}}{\pi}$ HINT: se på $x^2$ siden 
$0\leq x\leq 2$ medfører at $x^2 \leq 4$.
\end{itemize}
\end{oppgave}

\begin{oppgave}
Begrunn hvorfor $x\in\la\leftarrow,5]$ er det samme som
at $x\leq 5$.
\end{oppgave}

\begin{oppgave}
Begrunn hvorfor $x\in\la\leftarrow,5 \ra$ er det samme som
at $x < 5$.
\end{oppgave}


\begin{oppgave}
Begrunn hvorfor $x\in\la\leftarrow,a \ra$ er det samme som
at $x < a$.
\end{oppgave}


\section{En motivasjon for algebra}

Når vi bruker matematikk til litt mer kompliserte ting enn å
gå på butikken, ønsker vi gjerne å spare arbeid. Derfor lager
vi regneregler for tallsystemene som skal fungere for \emph{alle}
disse tallene. Vi kan si at algebraen i skolen er å bruke
disse reglene til å manipulere \emph{uttrykk}. La oss ta et eksempel!

\begin{eksempel}\label{utsagn1}
Vi skal bruke reglene fra algebra til å vise at
\begin{align} 
\frac{(xy)^3x^4}{yx^2}  = y^2x^5 \text{ for } x,y \neq 0
\end{align}

Først \emph{distribuerer} vi ut potensen;
\[ \frac{(yx)^3x^4}{yx^2} = \frac{x^3y^3x^4}{yx^2} \] 
Deretter bruker vi at \emph{multiplikasjon er en kommutativ operasjon}, 
         det vil si at rekkefølgen på faktorene ikke spiller noen rolle:
\[ \frac{y^3x^3x^4}{yx^2}\]
Deretter slår vi sammen potenser av like grunntall:
\[ \frac{y^3x^{3 + 4}}{yx^2} = \frac{y^3x^{7}}{yx^2}\]
Nå \emph{forkorter} vi brøken
\[ \frac{y^3x^7}{yx^2} = \left(\frac{y^3}{y}\right)
    \left(\frac{x^7}{x^2}\right) = y^2x^5\]
\end{eksempel}


Vi kan si at likningen (\ref{utsagn1}) er et \emph{utsagn} som sier at om vi tar 
hvilke som helst reelle tall $x, y \neq 0$ og setter inn i både høyre og venstre
side, så skal svaret \emph{alltid} bli det samme.
Slike utsagn er også svært nyttige i anvendelser, da bruker man matematikken
til å beskrive generelle sammenhenger. Et berømt eksempel fra fysikken er 
Newtons lover, som sier:
\begin{itemize}
\item[I)]
Et legeme som ikke er utsatt for ytre krefter vil bevege seg i en
rett linje

\item[II)]
Sammenhengen mellom ytre krefter $F$ på et legeme med masse $m$ og dets
akselerasjon $a$ er gitt ved
\[ F = ma .\]

\item[III)]
Envher kraft har en motkraft, som er like stor og peker i motsatt retning

\end{itemize}



Vi ser spesielt at NII (Newtons andre lov) gir oss en generell sammenheng
mellom kraft og akselerasjon. Denne enkle loven er grunnlaget for å sende folk
til månen, all mekanisk ingeniørvitenskap som dekker alt fra bevegelse av
væsker og gasser (i atmosfæren - meteorologi, i rør -
        oljeindustri/skipsfart/industrielle anlegg, beregning av løft på fly
        eller tak under orkan...) til bevegelseslikningene for faste stoffer, samt for å forstå
knekning og bøyning. Man kan altså ramse opp det meste av moderne teknologi med
utspring i denne likningen. Den er også viktig for å forstå fenomener i
elektronikk, da man der også analyserer krefter på ladde partikler med masse
\footnote{man ser da bort fra relativistiske/kvantefysiske effekter}.

\begin{eksempel} 
Kristoffer dytter på en kloss med masse 2kg langs en rett linje, med
kraft 2N. Akselerasjonen er da \[a = \frac{F}{m} = \frac{2N}{2kg} = 1N/kg = 1
\frac{m}{s^2}\]
Altså øker hastigheten med $1\frac{meter}{sekund}$, per sekund.
$N$ er enheten Newton, og den har de fundamentale enhentene $\frac{kg m}{s^2}$.
\end{eksempel} 

Du vil altså se i fysikken at å regne med bokstaver er mye kraftigere enn
å regne med tall, og det er også mye ryddigere! Som i newtons lover
vet vi da hvilken fysisk størelse hvert symbol representerer, og det
blir fort rotete om man setter inn tall altfor tidlig. Dette fordrer imidlertid
at vi har klare regler for hvordan tall oppfører seg, og at vi er inneforstått
med reglenes \emph{gyldighetsområde}, det vil si hvilke situasjoner de gjelder
for.

\section{Utvidelse av brøk}
Vi skal oppsummere noen regler for brøk og potenser.

\begin{aksiom}
For alle tall $x\in \R$ som ikke er null, finnes et \emph{unikt}
tall \[\frac{1}{x} \in \R \]
slik at \[ x\left(\frac{1}{x}\right) = 1.\]
\end{aksiom}
Dette kan virke litt abstrakt\footnote{
    Her bør det nevnes at et \emph{aksiom} er en forutsetning vi antar at
        tallsystemet har, uten å begrunne det videre.}
        , men det er denne
regelen som lar oss forkorte brøker. Om vi deler et tall på
seg selv, får vi altså $1$ tilbake. Den sier også at dette er
det eneste tallet som gjør jobben. Merk videre at vi her har
definert divisjon som multiplikasjon med en \emph{multiplikativ invers}.
Utrykket $a(\frac{1}{x})$ skrives ofte som
\[\frac{a}{x}\]

Videre er det verdt å merke seg at siden vi har 
\[\frac{x}{x} = 1 \] kan vi fritt også \emph{utvide}
brøker for å f.eks addere to brøker.
\begin{eksempel}[Utvidelse og addisjon av brøk]
Vi skal slå sammen brøkene \[\frac{a}{b} + \frac{c}{d}\]
Vi må da utvide brøkene slik at vi får like nevnere. Vi gjør
følgende:

\begin{align}\label{brokutvidelse}
\frac{a}{b} + \frac{c}{d} = 
\left(\frac{a}{b}\right)\frac{d}{d} + \left(\frac{c}{d}\right)\frac{b}{b}
        = \frac{ad}{bd} + \frac{cb}{db} = \frac{ad + cb}{db}
\end{align}
Der vi i første likhetstegn kun ganger begge brøkene med $1$, og i andre
likhetstegn bruker vi at 
\[\frac{a}{b} = a\frac{1}{b},\;\; \frac{d}{d} = \frac{1}{d}d\]
så
\[\frac{a}{b}\frac{d}{d} = a\frac{1}{b}\frac{1}{d}d\]
Utrykket i midten skriver vi \[\frac{1}{bd}.\]
Da dette ganget med $bd$  blir $1$ (overbevis deg selv om dette
        ved å gange inn $bd$ og se at du får $1\cdot 1 = 1$).

\end{eksempel}

\subsection{Oppgaver}
\begin{oppgave}
Dersom du behøver litt trening på faktorisering ved bruk av
det distributivet aksiom $a(b + c) = ab + ac$ kan du faktorisere
noen av uttrykkene her: (skrive dem på formen $a(b + c)$
\begin{itemize}
\item[i)]
    $3x^5 + 21x^3$
\item[ii)]
    $3x^5 + 21x^3$
\item[iii)]
    $x^{2009} + 5yx^5$
\item[iv)]
    $(5x)^{2014} + (5x)^{2020}$
\item[v)]
    $(5x)^{2014} + (10x)^{2020}$
\item[vi)]
    faktoriser alle mulige ledd på formen $u(v + w)$ \newline
    $3x^5 + 21x^3 + 10x^5y^3 + abc + 5yx^{19} + \nu$
\end{itemize}
\end{oppgave}

\begin{oppgave}
En vanlig feil mange gjør er å slå sammen brøkene uten å utvide:
\[\frac{a}{b} + \frac{c}{d} \] blir til \[\frac{a + c}{b + d} \]
Finn tall $a,b,c,d$  slik at
\[\frac{a}{b} + \frac{c}{d} \neq \frac{a + c}{b + d} .\]
\end{oppgave}

\begin{oppgave}
Begrunn siste steget i liking (\ref{brokutvidelse}) med å bruke
\[\frac{a}{b} = a\frac{1}{b}\] og at \[a(b + c) = ab + ac.\]
(Den siste av disse er forøvrig det \emph{distributive} aksiomet for
 reelle tall).
\end{oppgave}

\begin{oppgave}
Utvid og skriv som en enkelt brøk.
\[\frac{7x^5}{6} + \frac{6y}{a},\]
der $a \neq 0$.
\end{oppgave}


\section{potensregler}
...

\subsection{oppgaver}


\begin{oppgave}
Utvid og skriv som en enkelt brøk.
\[\frac{7x}{6} + \frac{6y}{x^{2}},\]
der $x \neq 0$.
\end{oppgave}


\begin{oppgave}
Utvid og skriv som en enkelt brøk.
\[\frac{7x^{13}}{6} + \frac{6y}{x^{2001}},\]
der $x \neq 0$.
\end{oppgave}


\begin{oppgave}
Anta at $n > m$ er to naturlige tall (altså heltall som er
        større enn null).
Begrunn at 
\[\frac{a^n}{a^m} = a^{n - m}.\]
HINT: Forkort brøken.
\end{oppgave}


\begin{oppgave}
Anta at $n$ og $m$ er to naturlige tall, men vi snur 
betingelsen om at $n$ er større enn $m$ fra forrige oppgave på
hodet og antar $n < m$ så $n - m < 0$
Begrunn at
\[\frac{a^n}{a^m} = \frac{1}{a^{m - n}} \]
HINT: Forkort brøken.

Vi lar dette være $a^{n - m}$, og derfor definerer vi
\[a^{-n} = \frac{1}{a^n} ,\] der $n$ er et naturlig tall.
\end{oppgave}

\begin{oppgave}
Vis at $(a^n)^m = a^{nm}$ for alle naturlige tall $n,m$.
HINT: Bruk definisjonen $a^n = a\cdot a\cdot a \cdots a$, der
$n$  er antall faktorer,
og tell opp hvor mange faktorer du får når du ganger dette med seg selv $m$
ganger.
\end{oppgave}


\begin{oppgave}
Begrunn at $a^{\frac{1}{2}} = \sqrt{a}$ for alle $a>0$ ved å bruke
formelen fra forrige oppgave.

Vi \emph{definerer} altså brøker på følgende måte;
$a^{\frac{m}{n}} = (^n\sqrt{a})^m$.
\end{oppgave}



\end{document}


